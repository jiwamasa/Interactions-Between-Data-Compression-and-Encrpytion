% ccmin.tex

\documentclass[11pt]{article}

\usepackage{fullpage}

% wider spacing
\renewcommand\baselinestretch{1.5}

% Use the postscript times font!
\usepackage{times}
\usepackage{latexsym}
\usepackage{graphicx}
\usepackage{float}
\usepackage{amsmath}

\usepackage{theorem}

%For graphs and trees
\usepackage{tikz}
\usetikzlibrary{automata, arrows, positioning}

\long\def\ignore#1\endig{}      % ignore (comment out) text thru \endig


%%%%%%%%%%%%%%%%%%% abbreviation-type macros %%%%%%%%%%%%%%%

\newcommand\genref{}
\newcommand\sectref{}
\newcommand\defnref{}
\newcommand\figref{}
\newcommand\thmref{}
\newcommand\eqnref{}
\newcommand\Eqnref{}
\newcommand\tblref{}

\def\genref#1#2{#1~\ref{#2}}
\def\sectref#1{\genref{Section}{#1}}
\def\defnref#1{\genref{Section}{#1}}
\def\figref#1{\genref{Figure}{#1}}
\def\thmref#1{\genref{Theorem}{#1}}
\def\eqnref#1{(\ref{#1})}
\def\Eqnref#1{Equation (\ref{#1})}
\def\tblref#1{\genref{Table}{#1}}

\newcommand\thin{\mskip\thinmuskip}

\newcommand\clause{}
\def\clause#1{\left[ #1 \right]}
\newcommand\set{}
\def\set#1{\left\{ #1 \right\}}

\newcommand\square{\raisebox{3pt}{\framebox[8pt]{\rule{0pt}{2pt}}}}

%%%%%%%%%%%%%%%%%%% END abbreviation-type macros %%%%%%%%%%%%%%%


% Specify numbering schemes
\theorembodyfont{\rmfamily}
\newtheorem{thm}{Theorem}[section]{\bfseries}{\rmfamily}
\newtheorem{defn}[thm]{Definition}{\bfseries}{\rmfamily}

% Allow larger tables and figures on same page with text.
\renewcommand\textfraction{0.25}
\renewcommand\dbltopfraction{1.05}
\renewcommand\topfraction{1.05}
\renewcommand\bottomfraction{1.05}



\begin{document}

\title{Interactions Between Data Compression and Encryption}
\author{James Iwamasa}
%\date{}
\maketitle

%\subsection*{Abstract}
\begin{abstract}
Every day, insurmountable amounts of data are shuffled around over an expansive web of users. 
While the infrastructure to facilitate such movements of information is nothing short of a world wonder, 
this system inherently poses many problems, the two biggest being that of bandwidth and security. 
While the fields of data compression and encryption are well developed and highly specialized in their 
own rights, they both involve the act of altering the raw data somehow, and as a result can come into 
conflict. This paper will explore these complex interactions between data compression and encryption.
\end{abstract}


\section{Introduction}\label{intro-sect}
The direction of this paper can be characterized by a single question: Do we compress our data before we 
encrypt or the other way around? Basic knowledge of standard encryption and compression techniques will 
imply that, typically, compression should come first. Data compression often works by noticing patterns and 
redundancies in our data. On the other hand, encryption attempts to make our data unrecognizable gibberish 
to anyone who doesn't know the secret code by removing all patterns. 
Thus, if we want our compression to have the most effect, we should not encrypt it first.

One may also have the intuition that compression might actually assist in encryption, 
since we're still converting our data into something only a computer can effectively decompress. 
But there have been studies\cite{kelsey, gluck} showing that compression can actually expose security flaws. 
In one method, hackers could use the difference in data length after compression to infer 
the plaintext of http requests (as in expoits CRIME and BREACH) \emph{without knowledge of the encryption technique.}

This conflict of interests between encryption and compression is the main crux of this paper. 
Now we will explore how the two can interfere with each other and also help each other in 
various contexts.


\section{Definitions}\label{def-sect}
First, we lay out some basic definitions:
\begin{defn}
\emph{Data compression} is the act of taking raw data and shrinking it to facilitate easier transport and storage. 
\end{defn}
\begin{defn}
A \emph{compression algorithm} performs the compression, while a \emph{decompression algorithm} returns 
compressed data to its original, uncompressed form.
\end{defn}
\begin{defn}
The \emph{compression rate} of some instance of compression is the amount, usually by percent, the algorithm 
shrank the data. 
\end{defn}
\begin{defn}
\emph{Data encryption} (in the field of \emph{cryptograpy}) is the act of taking raw data and transforming it 
such that only the intended end users can use it.
\end{defn}
\begin{defn}
We \emph{encrypt} data into its encrypted form, and the recipient \emph{decrypts} it to get the original data. 
\end{defn}
\begin{defn}
A \emph{key} is some method or string that is paired (in optimal cases, uniquely) with an encoded set of data or 
an encoding method. The key is then used to decode the data, the assumption being that one could not decode the 
data without knowing the key.
\end{defn}

\section{A Basic Example}\label{simple-example-sect}
We start by analyzing the interactions between compression and encryption with an example:\\
Let us say we have some data string $s$ which is a string of capital alphabetical letters A-Z. 
We shall now apply both a compression algorithm and an encryption method onto $s$ and observe what happens.  
We will use \emph{Huffman encoding}\cite{huffman} as our method of compression, as it embodies many of the 
major elements of data compression, and a simple \emph{block cipher} for our encryption for the same reasons.

\subsection{Huffman encoding}\label{huffman-subsect}
Huffman encoding involves assigning symbols variable length bit representations based on their relative 
frequency in the string. The general idea is that more common symbols get shorter representations, 
and less frequent ones are given longer ones. In the case of our alphabet, we would need at least 5 
bits per symbol to represent them all by a simple enumeration (A = 00001, B = 00010, etc). So Huffman encoding's 
benefit is twofold: Both by only using the number of bit representations we need, 
and by using the symbol frequencies to greatly reduce the space used by common symbols.\\
Let's first consider string $s$:
\begin{center}ABRACADABRA\end{center}
First we find the relative frequencies of our symbols, which we put in this table:
\begin{figure}[H]\begin{center}\begin{tabular}{ r | l }
	Symbol & Frequency \\
	\hline
	A & 0.45 \\
	B & 0.18 \\
	C & 0.09 \\
	D & 0.09 \\
	R & 0.18 \\
\end{tabular}\caption{Table of symbol frequencies in "ABRACADABRA"}\end{center}\end{figure}
Next, we must construct our Huffman encoding tree, which is a binary tree that we will use to assign bit representations 
and also to decode our Huffman encoded string. First, we convert all our symbol/frequency pairs into binary tree nodes 
with a symbol label and a frequency value and put them in a container $A$. Then, we will go into a loop, constructing 
our tree until there is only one element in $A$:
\begin{itemize}
	\item[1.] Find the two nodes $a$ and $b$ in $A$ whose frequency values are the lowest, and remove them from $A$.
	\item[2.] Create a new node $\alpha$ whose children are $a$ and $b$ and whose value is the sum of their frequency values. 
	\item[3.] Add $\alpha$ to $A$ and go back to step 1. 
\end{itemize}
For $s$, our tree will look like this, where greek letters are used to represent constructed nodes:
\begin{figure}[H]\begin{center}\begin{tikzpicture} 
	\tikzset{vertex/.style = {shape=circle,draw,minimum size=1.5em}}
	\tikzset{edge/.style = {->,> = latex'}}
	\node[state] (a) at (0,0) {};
	\node[state,below right = of a] (b) {$\gamma$, 0.54};
	\node[state,below left = of a] (c) {A, 0.45};
	\node[state,below left =10em of b] (d) {$\beta$, 0.36};
	\node[state,below right =10em of b] (e) {$\alpha$, 0.18};
	\node[state,below right = of d] (f) {R, 0.18};
	\node[state,below left = of d] (g) {B, 0.18};
	\node[state,below right = of e] (h) {D, 0.09};
	\node[state,below left = of e] (i) {C, 0.09};
	\path[]
	(a) [->]edge node[above left] {0} (c)
	(a) [->]edge node[auto] {1} (b)
	(b) [->]edge node[above left] {0} (d)
	(b) [->]edge node[above right] {1} (e)
	(e) [->]edge node[auto] {0} (h)
	(e) [->]edge node[above left] {1} (i)
	(d) [->]edge node[auto] {0} (f)
	(d) [->]edge node[above left] {1} (g)
	;
\end{tikzpicture}\caption{Huffman encoding tree for "ABRACADABRA"}\end{center}\end{figure}
\begin{figure}[H]\label{huffman-table-fig}\begin{center}\begin{tabular}{ r | r | l }
	Symbol & Frequency & Bit code\\
	\hline
	A & 0.45 & 0\\
	B & 0.18 & 101\\
	C & 0.09 & 111\\
	D & 0.09 & 110\\
	R & 0.18 & 100\\
\end{tabular}\caption{Table with bit representations of symbols in "ABRACADABRA"}\end{center}\end{figure}
Note that each edge in our tree is weighted with either 0 or 1.
As shown in \figref{huffman-table-fig}, we get the bit representations of our symbols from the tree by traversing the 
tree, keeping track of which edge we go down, until we hit a leaf node. So our encoded string $H(s)$ is:
\begin{center}0.101.100.0.111.0.110.0.101.100.0\end{center}
Which is 23 bits in total, compared to the 55 bits it would have taken if each symbol took a 5 bit int representation 
per symbol. To decode this, we start at the beginning of $H(s)$ and the root of the tree, reading in symbols 
and traversing the tree accordingly. When we hit a leaf node, we write the symbol on that node to output, 
go back to the root of the tree, and resume reading $H(s)$.

Huffman encoding, while not usually used by itself in practice, is a very fundamental and representative compression 
algorithm for two main reasons: It avoids the problem of unused space, using only the bits we need, and 
uses the patterns in the input (in this case, the symbol frequencies) to further reduce the amount of information needed.

One may notice that Huffman encoding acts similarly to a encryption algorithm. To decode a Huffman encoded string, 
one requires the Huffman encoding tree used to encode it, which thus acts like a key. Indeed, one paper\cite{sangwan} explored 
a combined compression/encryption technique where the data is first compressed with Huffman encoding, and the 
resulting tree is then encrypted with a secret key. It is then sufficient to not encrypt the compressed data, 
as it would be difficult to decode without the tree.

\subsection{Block cipher}\label{block-cipher-subsect}
Now let us study block ciphers, and how we use them to encrypt data. A normal cipher (or "stream cipher") involves 
reading our data string $s$ and replacing the characters one by one to encode our string. 
ROT13, or the Caesar cipher, is a commonly taught stream cipher where each character in our string is 
simply shifted by 13 letters. Block ciphers work in a similar manner, only instead of operating on single 
characters, we operate on blocks of some fixed $n$ characters at once. 
A block cipher is made of two main elements:
\begin{itemize}
	\item[1.] A fixed block size $n$.
	\item[2.] A \emph{round function}.
\end{itemize}
To encrypt our string, we break up the string into blocks of size $n$, and then apply our round function to each block. 
A round function is usually a collection of simple operations, some of which include:
\begin{itemize}
	\item[1.] Modulus addition: Adding some key value to each block (like with the ROT13 cipher).
	\item[2.] Rotations: Shifting elements in the string, wrapping the ends of the block back around.
	\item[3.] XOR adjacent blocks: After applying some other operations, XOR-ing the current block with, 
for example, the previous one. 
\end{itemize}
As implied by the "round" aspect, we may apply any of these multiple times to increase encryption at the cost 
of runtime efficiency. Decryption is then just running one's particular sequence of operations in reverse on the 
encoded string.

For our simple cipher $C()$, we will use a modular addition method similar to ROT13, but using a key instead 
of a flat, universal constant. We define $C(s)$ with block size $n$ and key $k$ where $|k|=n$ as follows: 
\begin{itemize}
	\item[1.] Assign numeric values 1-26 to each letter (A = 1, B = 2, etc).
	\item[2.] Divide $s$ into blocks of size $n$.
	\item[3.] Add $k$ to each block (that is, add the numeric value of the first character in $k$ to the 
numeric value of the first character in the block, wrapping around the alphabet, and so on). 
\end{itemize}
Let us perform our cipher on "ABRACADABRA" with $n=4$ and $k=$ "MAGE":
\begin{center}
	ABRA.CADA.BRA $\rightarrow$ 1 2 18 1 . 3 1 4 1 . 2 18 1\\
	$+$ MAGE $\rightarrow$ 13 1 7 5\\
	14 3 25 6 . 16 2 11 6 . 15 19 8 $\rightarrow$ NCYF.PBKF.OSH\\
\end{center}
Thus our string is encrypted. To decrypt using our simple cipher, we simply break up the encrypted string into 
blocks again, and subtract off the key from each block.

Note that the letters in the encrypted version of the string are much more randomly distributed than in the 
original string (which we will quantify in \sectref{analysis-subsect}). This is a good property to have for encryption, 
as obvious patterns may reveal too much information even without knowledge of the key.

\subsection{Combining encryption and compression}\label{encrypt-plus-compress-subsect}
Now that we have seen what these methods do on their own, let's see what happens when we apply these two methods together. 

\subsubsection{Compression then encryption}\label{e-then-c-subsubsect}
Let us first try the canonical method of compression before encryption. Using the result of our Huffman encoding, we will compute 
$C(s)$ with block size $n=5$ and $k=$ "10101". Note that our modular addition here will be just a bitwise XOR operation between 
$k$ and the block:
\begin{center}
	01011.00011.10110.01011.000 $\rightarrow$ 11110.10110.00011.11110.101\\
\end{center}
This works just as we would expect, as our cipher is not directly affected by the content of the string. Our string is 
safely encrypted, and still retains the same compression rate from before. Moreover, encrypting after compression 
is generally faster since we have fewer symbols to encrypt. 

\subsubsection{Encryption then compression}\label{c-then-e-subsubsect}
We now try performing the compression after we have encrypted the data. Using the encrypted string we got earlier, we find 
our frequencies and construct our Huffman encoding tree:
\begin{figure}[H]\label{huffman-table2-fig}\begin{center}\begin{tabular}{ r | r | l }
	Symbol & Frequency & Bit code\\
	\hline
	B & 0.09 & 111\\
	C & 0.09 & 110\\
	F & 0.18 & 000\\
	H & 0.09 & 101\\
	K & 0.09 & 100\\
	N & 0.09 & 0111\\
	O & 0.09 & 0110\\
	P & 0.09 & 0101\\
	S & 0.09 & 0100\\
	Y & 0.09 & 001\\
\end{tabular}\caption{Table with bit representations of symbols in "NCYFPBKFOSH"}\end{center}\end{figure}
\begin{figure}[H]\begin{center}\begin{tikzpicture} 
	\tikzset{vertex/.style = {shape=circle,draw,minimum size=1.5em}}
	\tikzset{edge/.style = {->,> = latex'}}
	\node[state] (a) at (0,0) {};
	\node[state,below right =8em of a] (b) {$\theta$, 0.63};
	\node[state,below left =18em and 10em of a] (c) {$\zeta$, 0.36};
	\node[state,below right =6em of c] (d) {$\beta$, 0.18};
	\node[state,below left =6em of c] (e) {$\alpha$, 0.18};
	\node[state,below right =2em of e] (f) {C, 0.09};
	\node[state,below left =2em of e] (g) {B, 0.09};
	\node[state,below right =2em of d] (h) {K, 0.09};
	\node[state,below left =2em of d] (i) {H, 0.09};
	\node[state,below right =6em of b] (j) {$\epsilon$, 0.27};
	\node[state,below left =1em and 4em of b] (k) {$\eta$, 0.36};
	\node[state,below right =10em and 4em of k] (l) {$\delta$, 0.18};
	\node[state,below left =2em of k] (m) {$\gamma$, 0.18};
	\node[state,below right =2em of m] (o) {O, 0.09};
	\node[state,below left =2em of m] (p) {N, 0.09};
	\node[state,below right =2em of l] (q) {S, 0.09};
	\node[state,below left =2em of l] (r) {P, 0.09};
	\node[state,below right =2em of j] (s) {F, 0.09};
	\node[state,below left =2em of j] (t) {Y, 0.09};
	\path[]
	(a) [->]edge node[auto] {0} (b)
	(a) [->]edge node[above left] {1} (c)
	(c) [->]edge node[auto] {0} (d)
	(c) [->]edge node[above left] {1} (e)
	(e) [->]edge node[auto] {0} (f)
	(e) [->]edge node[above left] {1} (g)
	(d) [->]edge node[auto] {0} (h)
	(d) [->]edge node[above left] {1} (i)
	(b) [->]edge node[auto] {0} (j)
	(b) [->]edge node[above left] {1} (k)
	(k) [->]edge node[auto] {0} (l)
	(k) [->]edge node[above left] {1} (m)
	(m) [->]edge node[auto] {0} (o)
	(m) [->]edge node[above left] {1} (p)
	(l) [->]edge node[auto] {0} (q)
	(l) [->]edge node[above left] {1} (r)
	(j) [->]edge node[auto] {0} (s)
	(j) [->]edge node[above left] {1} (t)
	;
\end{tikzpicture}\caption{Huffman encoding tree for "NCYFPBKFOSH"}\end{center}\end{figure}
Thus, we get the compressed data $H(C(s))$:
\begin{center}0111.110.001.000.0101.111.100.000.0110.0100.101\end{center}
Which is 37 bits long, significantly longer than the string we got as a result of compressing first (which was 23 bits, 
compared to the normal representation of 55 bits). The reason for this 
can be easily seen in the Huffman encoding tree we constructed. Huffman encoding assigns bit representations 
based on relative frequency. When we encrypted with our cipher, the frequencies were scrambled in an evenly distributed way, 
which, while good for encryption, is not good for compression. So we end up with a much larger tree which, in itself, also 
adds some overhead, since we need to communicate the tree to the end user so that they can decompress the data.

\subsection{Analysis through entropy}\label{analysis-subsect}
We can analyze this effect quantitatively through the use of a concept known as \emph{entropy}.
Defined by Claude Shannon in his paper "A Mathematical Theory of Communication"\cite{shannon}, entropy is the measure of 
the unpredictability of some event, which implies the amount of information we need to accurately represent the event. 
If our entropy is 0, then that means we know the outcome of the event without any information (ex. flipping 
a coin with both sides being heads). But as our entropy gets higher, our events require more information to predict. 
Mathematically, Shannon defines the entropy $H$ of a set of $n$ events whose probabilities are members of 
the set $\{p_1,p_2,...,p_n\}$ as:
	$$H=-K\sum_{i=1}^{n}p_i\log{p_i}$$
Where $K$ is some constant. In the context of our problem, our events are whether a specific symbol comes up 
with some probability $p_i$ and we use a base of 2 to put our measure of entropy in terms of binary bits.

The conflict between encryption and compression can be summarized thusly: compression typically depends on 
the original input having low entropy while encryption tends to increase entropy. To show the formor point, let's
look at the case where our input consists of $n$ symbols that all appear equally (with probability = $\frac{1}{n}$. 
Our entropy for this input is:
	$$H=-\sum_{i=1}^{n}\frac{1}{n}\log{\frac{1}{n}}=\sum_{i=1}^{n}\frac{1}{n}\log{n}=\log{n}$$
Note that this result, $\log{n}$, is the maximum entropy for a particular set of $n$ symbols. What happens when we apply our 
Huffman encoding scheme to this input? By our definition of the algorithm:
\begin{itemize}
	\item[1.] First, we run through our set of nodes and pair all of them into sub-trees of 3 nodes. 
At this point, our set of nodes contains all the parents of the symbols, and they all (in the case where the number 
of symbols is even) have the same probability. The number of nodes in our set at this point is then $\frac{n}{2}$.
	\item[2.] We repeat step 1 on the nodes in our set, creating another level of nodes, all with the same probability.
	\item[3.] We end the algorithm when only one node remains, so we end up with a total of $\log n$ levels 
in our tree, giving each symbol a bit representation $\log n$ bits long. 
\end{itemize}
The inefficiency of this encoding is apparent if we consider an input alphabet of 128 symbols, that is, the ASCII alphabet. 
If the frequencies of our symbols were all equal ($p_i=\frac{1}{128}$) as we assumed, our Huffman encoding would give each 
ASCII symbol a bit representation $\log128 = 7$ bits long, which is no better than 
the standard ASCII encoding: exactly what we were trying to avoid by using compression!

This observation is validated by Shannon's Source coding theorem\cite{shannon} which states that the average 
length of the codewords (in bits) 
for some input cannot go lower than the entropy of the original input without some definite loss of information. 
This applies for all encoding methods, not just Huffman encoding (although it should be mentiond that Huffman encoding 
does tend to get very close to the theoretical limit in terms of minimum bit length\footnote{ 
In general, Huffman encoding works better as the individual symbol frequencies reach some inverse of a power of 2 
to avoid wasted space from "fractional bits". Other methods try to solve this by encoding groups of symbols rather than 
just individual symbols.}). 
So the lower our entropy, the smaller our codewords get on average, resulting in a greater level of compression.

A good encryption algorithm, on the other hand, raises entropy. We can see why by looking at the simple Caesar cipher 
we talked about previously. Again, the Caesar ciper encrypts a string symbol by symbol by shifting each symbol 
in its alphabet by some constant (the canonical example being by 13 for symbols in the English alphabet). 
Let's say our string $s$ that we want to encode is a sufficiently large block of natural English text. We will keep our 
encryption method simple, and say that our key $k$ is the amount by which we shift.

First we note that applying our Caesar cipher does NOT increase the entropy of our string; 
it merely rearranges the probabilities of the symbols. What this exposes is the fact that, in normal English text, 
letter frequencies are well defined, allowing a potential hacker to potentially be able to decrypt 
the string without even knowing the key. This hack can be further improved by the fact that certain letters come 
up more often after others ("th", for example, is very common compared to "qx") as well as other various syntactic 
and semantic patterns. By raising entropy, we essentially make the frequencies of each symbol closer to each other, 
making it difficult to find such patterns.

So how does our simple modulus addition block cipher hold up to this model? Let's say we have a block cipher $C()$ with 
block size $n$ and key $k$ which is some string of length $n$. Let's also assume our string $s$ uses symbols from 
an alphabet of size $L$, and that our original symbol frequencies $=\{p_1,p_2,...,p_L\}$. We also add that 
the choice of our key is totally random (not something silly like "PASSWORD").

First, let's look at the ideal case (we will discuss why this is ideal later) where $|s|=n$, 
in which the "block frequencies" (the relative symbol frequencies within a block) are equal to 
the symbol frequencies of the whole input. The probability that some symbol $x_i$ occurs in 
our encrypted string $C(s)$ is then completely random, since we assumed our key is random. 
So, our new frquencies are just:
	$$p'_i=\frac{1}{L}$$
Plugging this into our forumla for entropy gives us:
	$$H'=-\sum_{i=1}^{n}\frac{1}{L}\log{\frac{1}{L}}=\log{L}$$
Which is the maximum entropy possible for this particular input. Thus, our Huffman encoding scheme 
would fail to compress this beyond what standard symbol encodings would give us.

This ideal case acts essentially the same as an encryption method called a "one-time-pad" cipher. 
Shannon theorized how one could achieve "perfect-secrecy" by having a key which uniquely and completely 
randomly transformed every individual lexical element of the message\cite{shannon-secrecy}. 
If done correctly, the encrypted message would be equivalent 
to completely random information, and should have a maximum possible entropy for that message. 
Our ideal modular arithmetic cipher applies a random shift to each symbol in the original string, making 
it effectively uncrackable without knowledge of the key.

Block ciphers in general (when $|s|$ is some multiple of $n$), however, are not nearly as strong.
Consider a block cipher with $k=$ "CAT". Now, consider encrypting this string with our cipher:
\begin{center}DOGDOGDOGDOG$\rightarrow$GPAGPAGPAGPA\end{center}
In this case, we can actually use our previous result to find the new entropy, 
since each block is identical (the frequencies would be the same if $s=$ "CAT"). But while the new entropy is 
the same, a vigilant hacker could realize the repeating pattern, infer the block size, and brute force search through 
keys of the length of the block size. We can find out how likely it is that this happens fairly easily. 
Given a string $s$, block size $n$ (for simplicity, assume that $|s|$ is a multiple of $n$), and alphabet of size $L$, 
the expected number of repeated blocks is:
	$$E=\sum_{i=2}^{|s|/n}(i){{|s|/n}\choose i}\Big(\Big(\frac{1}{L}\Big)^n\Big)^i\Big(1-\Big(\frac{1}{L}\Big)^n\Big)^{(|s|/n)-i}
	=\Big(\frac{1}{L}\Big)^n\Big(\frac{|s|}{n}\Big)\Big[1-\Big(1-\Big(\frac{1}{L}\Big)^{n}\Big)^{(|s|/n)-1}\Big]$$
This problem can be generalized to the problem of "how many times will a group of  
$t$ symbols come up multiple times in the same positions in different blocks?", as strings like 
"\underline{DOG}FOOBAR\underline{DOG}" and "D\underline{O}GC\underline{O}DR\underline{O}MS\underline{O}W" 
are also somewhat vulnerable. Our general formula is thus:
	$$E=(n+1-t)
	\sum_{i=2}^{|s|/n}(i){{|s|/n}\choose i}
	\Big(\Big(\frac{1}{L}\Big)^{t}\Big)^i
	\Big(1-\Big(\frac{1}{L}\Big)^{t}\Big)^{(|s|/n)-i}$$
	$$=(n+1-t)
	\Big(\frac{1}{L}\Big)^{t}\Big(\frac{|s|}{n}\Big)
	\Big[1-\Big(1-\Big(\frac{1}{L}\Big)^{t}\Big)^{(|s|/n)-1}\Big]$$
As $n$ approaches $|s|$ (approaching our ideal case), or equivalently, as the total number of blocks in our string goes down, 
we see that the number of repeats goes down (which is intuitive, since the more blocks we have the more 
chances we have to see repeats). So to reduce the chance of repeats occuring, we would like to increase 
our block size as much as possible. 

But note that repeats and block size don't have as a direct sway in the entropy of the data. 
In the case of something like the block cipher, we actually need to consider the entropy of our keys rather than the data, 
since, as we saw with our first example with "ABRACADABRA", our strings do end up fairly random in general. 
A block cipher is very vulnerable when the block size is known, and finding out the block size can be 
done by finding repeated patterns (since all blocks are operated on the same way, similar to why our Caesar cipher fails). 
For a key of length $b$ bits, we only have $b$ bits of entropy at most, and since it's inefficient to have a key 
of any signifcant length comparable to the input, simple block ciphers like the one we used are typically too weak 
to use on their own. However, if we compress our data first, our key becomes larger compared to the size of the data, 
thus increasing the strength of our algorithm, further pushing our canonical paradigm. 

\subsection{A quick runtime perspective}\label{runtime-subsect}
As mentioned in \sectref{block-cipher-subsect}, encryption is done by applying a round function of several simple operations 
on a string in multiple rounds. We can thus increase the strength of our encryption almost arbitrarily 
just by applying more rounds, but this has a serious runtime cost. Especially if you have a large amount of sensitive data to 
encrypt, the speed of your encryption can become just as important as its effectiveness. Likewise, the speeds of our 
compression algorithms are just as much a consideration as the compression rate.

In our example, the speed of our 
Huffman encoding algorithm was greatly slowed down when applied after the encryption (as can be seen in the difference in 
sized of the Huffman encoding trees). The encryption in this case was actually faster because of an abstraction we 
made that allowed us to perform modular addition directly on the symbols themselves rather than their bit representations. 
However, in a practical scenario we should have stayed consistent, and reconverted the compressed bit string 
representation into whatever characters they would have 
represented in, say, ASCII. In this case, and in general, it should be obvious that it's faster to apply 
an encryption algorithm on a shorter, compressed string rather than an uncompressed one. 

\section{Non-canonical methodologies}\label{non-canon-sect}
The previous section gave us a simple example showing us why we generally compress before encryption. 
In our example, not only was the compression much more effective, but overall runtime was improved. 
However, this can actually expose security flaws. 

\subsection{CRIME and BREACH: The Reason}\label{crime-breach-subsect}
CRIME and BREACH\cite{kelsey, gluck} are two recent security exploits discovered where hackers use the 
difference in length from compression to infer plaintext from http responses. As a result, even without 
knowledge of the encryption method, one could find out information about the response.

Here's the basic idea: 
Let's say there is some secret key $k$ within a sequence of http requests and responses that 
you want to find out. Most http responses are compressed by the same algorithms, one of the 
main compression methods being a so called "dictionary" encoding scheme, where one takes sequences 
of characters that are the same, and replaces them all with references to a single instance of that string. 
Say our response contains the string "What is that hat?". This form of encoding would convert that to: 
"W\%1 is t\%1 \%1?", where "\%1" is a reference to the string "hat". 

So how can we use this to find $k$? Let's say we can isolate the key from the rest of the response. 
What we then do is inject either short strings of characters or single characters into the responses, and see how long 
the compressed response is. The information we get is then how much our "guess" matched up with the actual key. 
For example, let's say that we make a lucky guess, and inject a near copy of $k$ into the response that has $k$ in it. 
We then have something like this, where the injected text is in bold:
\begin{center}
	Original response: my\_key=password \textbf{my\_injection=passwort}\\
	Length = 39\\
	Compressed response: my\_key=\%1d \textbf{my\_injection=\%1t} (\%1=passwor)\\
	Length = 29\\
\end{center}
Now, we try injecting another string slightly different from our previous injection:
\begin{center}
	Original response: my\_key=password \textbf{my\_injection=password}\\
	Length = 39\\
	Compressed response: my\_key=\%1 \textbf{my\_injection=\%1} (\%1=password)\\
	Length = 27\\
\end{center}
Note that we could have cycled through all possible characters for the last symbol in our injection, 
and we would know when we got the right one just by the difference in size from compression (from 29 to 27). 
In general, this method allows us to build up character by character hidden secrets in http responses 
without even knowing the encryption algorithm. 

\subsection{Compressing encrypted data}\label{compress-encrypt-subsect}
While there are specific counter-measures for CRIME and BREACH (not using http compression, special encryption, etc), 
the issue of information leakage through compression can potentially spread to other means to communications. 
To solve this problem, we will now look at a method of efficiently compressing data after encryption as described in \cite{johnson}. 

As shown in \sectref{analysis-subsect}, compression is limited inversely with the entropy of our data. 
Since any good encryption algorithm will typically raise the entropy of our data to the theoretical limit, 
by the aforementioned Shannon's source coding theorem, we cannot create codewords to compress 
our data by any amount. While the method performed by \cite{johnson} does not actually work around 
this theoretical limit, it cleverly uses the key from the encryption itself as a "key" to help compress and 
decompress our data. 

\subsubsection{Low-Density Parity-Check Codes}\label{ldpc-subsubsect}
One part of this method stems from a concept known as linear codes, in particular, 
low-density parity-check (or LDPC) codes[CITE GALLAGER].
These codes, interestingly enough, are actually more often used to \emph{increase} the size of data for the purpose of 
making our data recoverable in the presence of noise (for example, satellite communications). 
Encoding some message $x$ with LDPC codes requires a few major elements: 
\begin{itemize}
	\item[1.]Integers $n$ and $k$ such that the length of the strings (blocks) we want to encode are of lengt $n-k$. 
The choice of $n$ and $k$ is dependant on the data itself, the noisiness of the channel, etc. 
	\item[2.]A \emph{parity-check matrix} $H$ which is $n\times (n-k)$.
\end{itemize}
First, we must calculate our \emph{generator matrix} $G$ which is achieved by transforming 
$H$ into the form:
\[ \left[
\begin{array}{c|c}
	-P^T & I_{n-k}
\end{array}
\right] \]
And then converting that into the form: 
\[ \left[
\begin{array}{c|c}
	I_k & P
\end{array}
\right] \]
Then, we break $x$ into blocks of size $n$ (labeled $b$), and our codewords are given as simply $bG$. 
Putting these codes back together gives us a new message $x'$, which, in most cases, is longer than 
our original string. 

We then send this over some noisy channel, where some of the bits become unrecoverable. 
To decode each noisy block $y'$ to the corresponding clean block $y=bG$, and then the corresponding original block $b$, 
we use a method called \emph{belief propagation}, 
where we iteratively try to figure out each missing bit using our parity-check matrix $H$. 
The algorithm works abstractly like this:
\begin{itemize}
	\item[1.]Treating $y'$ as a row vector, find a column $c$ with an unknown bit in $y'$ which satisfies these prerequisites:
	\begin{itemize}
		\item[i.]There exists at least one row in $H$ whose $c$th bit is 1.
		\item[ii.]Of those rows, wherever there is a 1 in some other column $d$, the $d$th bit of $y'$ is known. 
	\end{itemize}
	\item[2.]If these are satisfied, then we can figure out the $c$th bit in $y'$ by the fact that all of the 
bits in our original $y$ follow the constraint that the sum of all bits in $y$ specified by a row in $H$ 
(in the same column as a 1 in that row in $H$) must be even (equivalent to mod 2). 
Since we know all ther other bits in the constraint, we can easily calculate the remaining unknown bit.
	\item[3.]Repeat 1 and 2 until all bits are found.
	\item[4.]Our original $b$ is gotten by taking the first $n-k$ bits of $y$.
\end{itemize}
In more general terms, our parity-check matrix $H$ acts as a set of constraints on the codewords we are 
allowed. Thus, even if some of the bits are lost, we can recover them based on the assumption that 
the codeword must have passed the constraints. 

Of course, one can see that the size of our codewords, if $k>n$ (which is generally the case), 
is much longer than the original block, which, 
while intuitively effective for buffering our data against losses, seems completely counter-effective for compression. 
We require one more method to make this effective. 

\subsubsection{Distributed Source Coding}\label{distributed-source-coding-subsubsect}
The next and last technique we require is related to the \emph{distributed source coding problem}\cite{johnson}. 
The problem involves trying to compress a message/stream $E$ with access to another stream $F$ 
that is correlated to it. The basic idea is that if we have $F$, then we can reduce the information 
needed to identify blocks from $E$. Or, using entropy, 
we see that the \emph{conditional entropy} of $E$ is less when we know $F$, 
allowing us to compress further. 

[DEFINE HAMMING DISTANCE]

We will now describe a simple solution to the distributed source coding problem: 
Let's say we have a message $E$ which will be broken up into blocks $b$ of size $n$. 
Let's also assume that the receiver of our message already has the correlated stream $F$. 
First we define a set of "cosets" of size 2 that consist of binary strings of length $n$ and 
whose elements have a hamming distance 

% \newpage

\bibliography{ec-bib}
% \bibliographystyle{splncs}
% \bibliographystyle{mod-aaai}
\bibliographystyle{alpha}

\end{document}

